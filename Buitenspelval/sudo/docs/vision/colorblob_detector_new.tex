\documentclass[a4paper, 10pt]{article}

\usepackage{fullpage}
\usepackage[english]{babel}
\usepackage{graphicx}
\usepackage{multirow}
\usepackage{array}
\usepackage{listings}
\usepackage{url}
\usepackage{hyperref}
\usepackage{courier}

\hypersetup{
	colorlinks=true,
	breaklinks=true,
	urlcolor= black,
	linkcolor= black,
    citecolor= black,
}

\lstset{language=c, frame=lines, basicstyle=\tt, tabsize=4, numberstyle=\tt}


\title{HOWTO - Start with the (New!) ColorBlob Detector \\ \small{Related courses: Robotica, PAS1 and PAS2}}
\author{Team BORG}

\begin{document}
\maketitle

\tableofcontents

\section{Introduction}

This document introduces you to the \textbf{(New!)} ColorBlob detector module.

\textbf{Please note that this document is a work in progress and may be updated in time.}

\section{Installation}

\begin{enumerate}
    \item Install python and opencv on a linux box:
\begin{lstlisting}
# sudo apt-get update
# sudo apt-get install python python-opencv 
\end{lstlisting}
    \item Use Python \lstinline{pip} or install manually if the above does not work.
\end{enumerate}

\section{Module Configuration}

An example configuration if shown below (should be on one line):
\begin{lstlisting}
    blobdetect = blobdetector_new  video_source=webcam trainable=True 
        targets=vision/colorblob/targets.pkl denoise_value=5 frequency=10
\end{lstlisting}

The following configuration parameters are specific for the blobdetector module:
\begin{enumerate}
    \item[\textbf{targets}:] The file as generated with the ``\lstinline{target_color_generate.py}'' script (see section \ref{sec:training}).
    \item[\textbf{denoise\_value}:] The amount of denoising required (0 = no denoise).
    \item[\textbf{frequency}:] The frequency of the module. The higher the frequency, the more observations are send to memory.
\end{enumerate}

In case of the Nao camera, you can use the following (should be on one line):
\begin{lstlisting}
    blobdetect = blobdetector_new  nao=<nao_ip> inputres=160x120 outputres=160x120 
        video_source=naovideo camera=1 trainable=True 
        targets=vision/colorblob/targets.pkl denoise_value=5 frequency=10
\end{lstlisting}

Set the camera value to 0 to use the upper camera, 1 to use the lower camera.

\section{Basic Training Procedure}
\label{sec:training}

First specify the colors to use using the ``\lstinline{target_color_generate.py}'' script\footnote{Choose 0 cubes if you do not want to train the module manually.}:
\begin{lstlisting}
# cd $BORG/brain/src
# python vision/colorblob/target_color_generator.py}
\end{lstlisting}

Once the colors are specified, you can train the module by setting the ``trainable'' configuration parameter to ``True'' and running it as shown in the ``example\_blobdetector'' behavior and configuration file.

\subsection{Keyboard/Mouse Shortcuts}

Once the module is running, you can cycle over all colors and train them using the following key and mouse shortcuts:

\begin{enumerate}
    \item[\textbf{Left-Mclk}:] Adds a new color cube for the selected color.
    \item[\textbf{Right-Mclk}:] Select the next colorblob.
    \item[\textbf{Mid-Mclk}:] Increases the size of the bounding box to sample the colors from.
    \item[\textbf{Shift Left-Mclk}:] Increases the Hue wideing.
    \item[\textbf{Ctrl Left-Mclk}:] Decreases the Hue wideing.
    \item[\textbf{Shift Mid-Mclk}:] Increases the Saturation wideing.
    \item[\textbf{Ctrl Mid-Mclk}:] Decreases the Saturation wideing.
    \item[\textbf{Shift Right-Mclk}:] Increases the Value wideing.
    \item[\textbf{Ctrl Right-Mclk}:] Decreases the Value wideing.
    \item[\textbf{Alt Left-Mclk}:] Saves the new configuration in the file specified in the configuration file.
\end{enumerate}

\section{Basic Usage (in Behavior)}

Once trained (see section \ref{sec:training}), you can retreive the detected observations in memory.
The topic corresponds to the names specified with the ``\lstinline{vision/colorblob/target_color_generator.py}'' script.

Each observation has the following fields:
\begin{enumerate}
    \item[\textbf{``width''}:] The width of the blob.
    \item[\textbf{``height''}:] The width of the blob.
    \item[\textbf{``x''}:] The X location of the upper left corner.
    \item[\textbf{``y''}:] The Y location of the upper left corner.
    \item[\textbf{``surface''}:] The area of the blob.
\end{enumerate}

\section{Training From a Behavior}

TODO

\section{Example Usage}

See the ``example\_blobdetector'' behavior and configuration file for more information.

\end{document}

