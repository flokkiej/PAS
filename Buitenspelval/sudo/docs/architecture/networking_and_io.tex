\documentclass[a4paper, 10pt, oneside]{article}
\usepackage[utf8x]{inputenc}
\usepackage{amsmath, amssymb}
\usepackage{graphicx}
\DeclareGraphicsExtensions{.pdf, .png}
\usepackage{verbatim}
\usepackage[colorlinks=true,urlcolor=blue,linkcolor=blue]{hyperref}
\usepackage{listings}
\usepackage{courier}

\title{Input and Output using TCP/IP sockets, Unix Domain Sockets, pipes or ptys}
\begin{document}
\maketitle

%set the font size of the code examples:
\lstset{ 
    basicstyle=\footnotesize,  
}

\tableofcontents

\section{Introduction}
This document describes the key elements in the communication infrastructure of
the Brain and the communicators. There are classes available to asynchronously
process IO requests on network and Unix Domain sockets, pipes and ptys. This
allows data to come in at any moment and be processed without any delay. The
VisionController uses this, for example, to put the observations into the memory
as soon as they come in, making them available sooner for any behavior waiting
for them. The key parts in this system are the use of polling mechanisms of
the operating system and a thread monitoring the IO objects(files, sockets,
pipes etc).

\section{The IOManager}
The IOManager class is a class that is used to manage objects that wrap file
descriptors in some way, such as network sockets, pipes, pty's and files.\\

The IOManager class should never be instantiated directly but should always be
accessed through the use of the IOManager() function that makes it sort-of
singleton. The reason the \_\_new\_\_ method is not used instead is that that
occasionally gives problems when registering new objects, due to the threaded
nature of the object.\\

The IOManager runs in a loop, polling all the registered file descriptors.  For
each file descriptor, functions are called to notify them that there is data to
read or write.\\

\subsection{Polling, Edge and Level Triggered mode}
For the polling, the IOPoll class is used. This is a wrapper around the
different polling mechanisms available on different platforms. These include
select, poll, epoll and kqueue/kevent. If the poller supports it, Edge Triggerd
behavior will be enabled on it. This also slightly changes the behavior of the
IOManager. If Edge Triggered behavior is enabled, for each file descriptor for
which an IO event is received, its handle\_read or handle\_write method will be
called continously until a IOWrapperEnd exception is caught. When this happens,
the file descriptor should not have any more data to read or write, to reset the
file descriptor in the epoll object. This implies that all reading and writing
on file descriptors in the handle\_read and handle\_write methods should be done
in a non-blocking way, otherwise the last call to those methods will block
instead of throwing the IOWrapperEnd exception. \\

The alternative to Edge Triggered mode is Level Triggered mode. The main
difference is that while Edge Triggered behavior only notifies the process of
new data or the becoming writable of the file descriptor, Level Triggered mode
will notify the process each time any file descriptor is readable or writable.
This results in much more events. When running in Level Triggered mode, after
each poll, the events are processed one by one. Each file descriptor that is
readable will be read from once and each file descriptor writible will be written
once. \\

See section \ref{iopoll} below for more information about polling.

\subsection{The Wake Up Pipe}
Access to the IOPoll object and the lists of registered IOWrappers is
synchronized using a lock. Mostly, the lock will acquired by the poll method of
the IOPoll object. This means that if a new wrapper is registered, or if new
data is available to write, it has to wait for the lock to be released before
the IOPoll object can be modified. To speed this up, a wake up pipe is
registered in IOPoll during startup. When another thread wants to acquire the
lock, it can write data to this pipe to wake up the call to poll, thus reducing
the waiting time and increasing performance. For this, the \_\_wakeup method is
used.

\subsection{IOWrapper Requirements}
Each file descriptor should be wrapper in a subclass of the IOWrapper class,
which enforces a set of methods to be available on the object.\\

In addition to the \_handle\_read and \_handle\_write methods, a
\_handle\_attempt and \_handle\_loop method should also be available. The
\_handle\_attempt is called for registered wrappper objects that are not
currently connected. This is useful mainly for network sockets that try to
connect to a remote host, or server sockets that cannot currently bind to their
port because the port is taken by another process or is otherwise unavailable.\\

The \_handle\_loop function is called around 10 times per second, and allows the
IOWrappers to perform some action on regular intervals.\\

Most importantly, each wrapper should define a fileno() method that returns the
file descriptor belonging to the object. This is used to register it in IOPoll
and to map any events to the correct wrapper.\\

Other functions that should be defined are close() to close the file descriptor
and unregister it from the IOManager, connected() that should return whether the
IOWrapper is connected/active (meaning that the fileno() method should return an
open file descriptor).\\

Current implementations of IOWrapper subclasses are ThreadedSocket and
ThreadedPipe. The first wraps a network socket, either client or server.  The
ThreadedPipe object wraps pipes in a non-blocking way.\\

Usually, you only need to concern yourself with IOManager when you are writing
or modifying a new IOWrapper, for example for reading files.

\subsection{Error Handling}
IOManager should be very stable because it will/can manage all IO of the system
and not just a single socket or file. Therefore, all calls to IOWrapper
instances are wrapped in a try/except block that unforgivingly removes the
wrapper from the IOManager if they result in an exception. It will generate a
CRITICAL message in the logger to inform the user about this. The only exception
is the IOWrapperEnd exception that should be raised by the \_handle\_read and
\_handle\_write methods of the IOWrapper class when there is no more data to read
or when no more data can be written.\\

Do take care that your subclasses/implementations do not generate exceptions or
the wrappers will be closed and removed from the wrapper.

\subsection{Profiling}
Sometimes it can be handy to profile the IOManager to see where delays in
network traffic are coming from. Under normal circumstances, most time should be
spent in the poll method from the IOPoll object; waiting for new IO to handle.
Profiling will enable profiling in the IOManager thread, so this will include
all function calls to the IOWrapper objects. This might indicate bottlenecks in
implementations.\\

To enable profiling, set the profile attribute of the IOManager class to the
filename to which to writing profiling information. For example:\\

\begin{lstlisting}
_IOManager.profile = "~/iomanager.prof"
\end{lstlisting}

This will store the profiling data to iomanager.prof in your home directory.
This profiling data can then be read using the pstats package. See the Python
documentation for more information, it's at:\\

\url{http://docs.python.org/library/profile.html}

\subsection{Stopping conditions: signals, timeouts and threads}
Any IOWrapper will at some point register itself with the IOManager by obtaining
an instance with the IOManager() function, that will instantiate and start a new
IOManager if needed. IOManager will also by default catch any signals the
process receives and react on those by shutting down the IOManager and closing
all the open file descriptors. Note that for this a shutdown flag is used to
avoid starting a new IOManager when signals have been caught. This makes sure no
new instance is started. If you want to handle signals in another way, you can
overwrite the signal handlers of IOManager and make sure the IOManager signal
handler is not called for that signal. If the IOManager is catching a signal and
you want to restart IOManager afterwards, you will have to set the shutdown flag
(iomanager.\_\_iom\_shutdown) to False afterwards.\\

When IOManager is running and no file descriptors have been registered for over
30 seconds, it will shut itself down to save processing cycles. As soon as any
IOWrapper tries to register itself with the IOManager, a new instance will be
registered.\\

IOManager also monitors the thread that started the IOManager in the first place
and will terminate as soon as that thread dies. This could have a undesired
effect if the IOManager is initially started from a thread that is not the main
thread of the program. If you are experiencing this in your code, a simple
solution is to initially start the IOManager from your main thread. You can
simply do this by putting the following code somewhere in your main thread,
before any IOWrapper is instantiated in any thread:\\

\begin{lstlisting}
from util.iomanager import IOManager
IOManager()
\end{lstlisting}

\section{IOPoll}
\label{iopoll}
IOPoll is a wrapper around several different polling mechanisms available on
different platforms. These mechanisms poll for events on file descriptors and
notify the user when a file descriptor is readable or writable.

Select is available on almost all platforms, but has severe performance issues,
especially with large sets of file descriptors. All file descriptors must be
passed on each poll using select. Therefore, select is available but used only
as a fallback for platforms on which no better alternative is available.

Linux provides the poll interface, with which a set of file descriptors can be
registered. These file descriptors will be polled each time the poll function is
called. Since the file descriptors are registered within the kernel, this
provides great performance benefits.

Since kernel 2.5.44, a new interface is available: epoll. This is a faster
version of poll and also allows the use of Edge Triggered polling. This means
that events are only reported once: if data is available to read, the poll will
return this once and then will not report this again until there is new data
available to read. This means that after the file descriptor has been reported
readable or writable, you have to exhaust the IO buffers and keep reading or
writing until all data has been read or written (and EAGAIN occurs), before new
events will be reported.

On BSD derivatives, kernel queues are available (kqueue). On these queues,
kernel events can be registered and these events will report readability or
writability of file descriptors.

To simplify the usage of the different polling mechanisms, the IOPoll class can
be used. Simply instantiate an IOPoll() object and the best available mechanism
is used. Using the poll method with a timeout in milliseconds, the respective
poll function of the selected mechanism is invoked and the events are returns in
a standardized format as either IO\_READ, IO\_WRITE or IO\_ERROR, or a combination
of those. Optionally, an argument specifiying a preference may be passed to
IOPoll which will try to use the specified polling mechanism. You can specify,
epoll, poll, kqueue or select, but the success depends on the availablility of
the mechanism.

\section{ThreadedSocket}
The ThreadedSocket is an instance of the IOWrapper for IOManager, to
network connections in an asynchronous manner.\\

\subsection{Comparable Systems}
The ThreadedSocket/IOManager combination shows some relation to the asyncore
system or the SocketServer framework but it has some major advantages. The use
of the IOManager allows allows the use of the IOPoll poller that can fall back
to the select class available on nearly every platform, the poll or epoll
interfaces on Linux, and the kqueue interface on BSD-derivatives. All these
interfaces perform significantly better than the select system call, especially
the Edge Triggered epoll mechanism on Linux. This will increase throughput and
responsivity of the IO manager. See the IOManager class for more information
about this.

\subsection{Functionality}
The system uses threads, hence the name. This means that while IO is processed
asynchronously, due to the Global Interpreter Lock on CPython, it still needs
some time from the interpreter to work once in a while. If your code uses busy
loops, the IOManager will not get the chance to process the incoming and 
outgoing data and therefore the performance might degrade.\\

When a ThreadedSocket object is instantiated, it will register itself with the
IOManager, as an unconnected socket, unless a connected socket has been passed
in, in which case it will be registered as a connected socket right away.\\

The IOManager will then call the \_do\_attempt function of the ThreadedSocket
approximately 10 times per second, at which the socket can try to connect. If it
actually does this 10 times per second depends on the retry\_timeout
parameter.\\

When the socket succesfully connects to the remote host or binds to the port, it
will register itself as a connected socket with the IOManager. As soon as there
are incoming connections or if there is data to read or write, the
\_handle\_read and \_handle\_write methods are called to handle this. Also, 10
times per second, the \_handle\_loop function will be called to perform any side
functions a subclass of the ThreadedSocket might want to perform. The processing
in this method should be minimal.\\

To allow subclasses to handle received data differently, the handle\_receive
method can be overridden. It will be called with the received data. The default
implementation will store it in the receive buffer which is a list of items
received in chronological order.

\subsection{IO: Socket Settings}
By default, sockets are opened in blocking mode and with the linger mode
disabled. This poses some problems. When running in Level Triggered mode,
blocking IO might be appropriate since the poll object knows for sure at any
moment that the socket is readable or writable when calling the respective
methods. But when using Edge Triggered mode, attempts to read and write will be
made until an error is returned. Therefore, the socket should be non-blocking to
avoid blocks when the kernel IO space is filled. For that reason, sockets
wrapped in the ThreadedSocket class are always set to non-blocking mode.
Additionally, for server sockets, problems may arise during debugging when a
reopening of the socket is attempted too soon after the previous socket was
closed. Because of the way TCP/IP is specified, a socket should be in a
TIME\_WAIT state for a while after it has been closed because it is more or less
guaranteed that the data will reach the host if it is still up. During the
TIME\_WAIT state, which can last from 2 to 4 minutes, usually no new socket can
bind to the same port. \\

There are two ways around this. The first is to allow binding to ports that
are already claimed by another socket. The other one is to avoid the TIME\_WAIT
state. The ThreadedSocket class uses the LINGER option to avoid the TIME\_WAIT
state. The close function on the socket will block for a little while to receive
more data and will then close. The reason REUSE\_ADDRESS is not used instead is
because that also allows reusing of sockets that are still actively in use, thus
giving problems when multiple servers are listening on the same port.

\subsection{Sending and receiving: cPickle and cStringIO}
The ThreadedSocket will use cPickle by default to wrap pickleable Python objects
before sending them over the socket. This allows for more complex structures,
such as dictionaries or NumPy arrays to be sent over the socket.\\

When the data has been preprocessed, it will be stored in a cStringIO buffer.
This buffer is used when sending the data. After each call to send, the number
of bytes actually sent determines the new position in the cStringIO buffer. This
way, the strings will not be copied every time but the internal pointer of the
cStringIO will be used. This gives much better performance.\\

When data is received, it is stored in a cStringIO buffer once again. This
buffer is then used by cPickle to load the pickled items one by one. When
unrecognized data appears, it is collected and added a separate string in the
resulting data. Pickled items could also be truncated because they have been
sent in several parts. When this happens, the data received so far is stored and
prepended the next time data arrives. Unpickling the data is then attempted
again. This way, no data should be lost.

\subsection{Unix Domain Sockets}
Unix Domain Sockets can be used on POSIX compliant operating systems. They
behave similar to normal sockets but bypass the TCP/IP stack, making them more
performant in almost all situations. The data does not have to be
wrapped and unwrapped in IP packets, which is already an improvement and because
communication over Unix Domain Sockets is always local, when a client writes
data to the server, it can immediately be stored in the right memory location.\\

As mentioned above, Unix Domain Sockets can only be used for communication
between process on the same machine. They are accessed through a file in the
filesystem but all communication takes place in the kernel. Therefore, mounting
a filesystem over SSH and attempting to use a domain socket that way will not
work.\\

The ThreadedSocket supports Unix Domain Sockets. To use them, provide a name as
a port argument instead of a port number. When the type of the port argument is
string, it is used as the name of a socket. The socket files created by
ThreadedSocket are stored in the /tmp directory. Thus, using port 'test' will
result in a socket file /tmp/test.\\

Because the files are accessed through the filesystem, one can easily remove or
restrict access to the socket by deleting the file or by changing the attributes
of the file. If yu remove the socket file, the socket will remain open but no
process will be able to connect to it.

\subsection{Ending a connection}
When connected to a ThreadedSocket, you can close the connection in several
ways. Sending a Ctrl+C or Ctrl+D sequence to the socket will result in closure
of the socket. This can be used when debugging using Telnet, for example.\\

Programmatically, you can just close the socket by closing the corresponding
file descriptor. ThreadedSocket will detect this at the other end and close its
own socket.

\section{BinarySocket}
The BinarySocket is an instance of the IOWrapper for IOManager, to handle
network connections in an asynchronous manner.\\

It uses the ThreadedSocket class to handle the IO related functionality of the
class.\\

The purpose of the BinarySocket class is to send binary data over a socket.
The pickle class can be used for this theoretically, but with large amounts of
data this is not efficient as pickling takes a lot of time. Each piece of binary
data can be accompanied by a piece of metadata. The metadata should be
relatively small as it will be pickled before being transported. The binary
string can be of any length. If you want, you can use the BinarySocket as a
replacement for threadedsocket as you can simply omit the binary data and just
send metadata; yielding more or less the same results, but this is not
recommended.\\

The BinarySocket works rather simply by prepending a 8-byte header before each
request. The first 4 bytes represent an unsigned integer giving the amount of
bytes of metadata in the current request, the next 4 bytes represent an unsigned
integer giving the amount of bytes of binary data in the current request. The
size of the metadata is calculated after pickling the metadata.\\

When receiving data, the header will be read. Afterwards, the specified amount
of bytes is read as metadata and the data is unpickled, and next, the specified
amount of bytes of binary data is read. When the entire request has been
received, the metadata and binary data is appended to the receive buffer and can
be obtained normally, as if you were retrieving data from the ThreadedSocket:\\

\begin{lstlisting}
data = binarysocket.get_data()
\end{lstlisting}

Here, data will be a list containing the received data. Each element will be a
tuple (metadata, binary\_data). You can then use the metadata to recreate
whatever the binary data representes. For example, you can store the depth,
number of channels and the dimensions of an opencv image in the metadata and use
the image.tostring() method to generate the binary data. This way, you can
transport images over the socket easily. This also works for, for example, large
numpy arrays, sound streams or other large data.\\

\subsection{Reliability}
The BinarySocket class is designed to be used over Unix Domain Sockets, TCP
sockets connecting to a local host or to a Local Area Network. It was not
designed to be used over a more unstable connection such as the internet. It
therefore does not split the data, use checksums or have means to refetch
incorrectly transferred parts, making the protocol more efficient over stable
connections. To transfer over the internet, more elaborate protocols are
required, for example FTP. 

\section{ThreadedPipe}
This class is an implementation of the IOWrapper base class, making it work with
the IOManager.\\

A ThreadedPipe wraps a pipe, returned by os.pipe() or another means of obtaining
a pipe. Basically, it will work with more pipe-like interfaces; such as pty's
and probably files as well. \\

Since there are lots of different methods to obtain a pipe, the constructor
itself will in most cases not create a pipe, but will accept the one passed in.
This can be an object that has a fileno() method, returning the file descriptor
of the pipe. It can also be an integer, the file descriptor itself. A third
option is to have the constructor create either a pipe or a pty automatically.
To achieve this, pass "pipe" or "pty" as argument to the constructor. Because
each pipe has another end and each pty has a slave, the slave file descriptor
can be obtained by calling the slave\_fd() method. This will return any slave
file descriptor that has been generated by the constructor.\\

IOManager requires non-blocking IO on all IOWrapper. Therefore, the pipe needs
tob e put in non-blocking mode. This cannot be done by the default Python
object, therefore we resort to the functionality provided by the fcntl module.
Also, because the file descriptor was provided from outside, there is a risk
that the file descriptor is closed by other means than through this wrapper. To
avoid these kind of problems, the file descriptor is duplicated with a call to
os.dup(). The new file descriptor will be independent of the original and thus
will remain open when the original file descriptor has been closed.

\subsection{Uses}
Currently, the ThreadedPipe is used in the GUI to monitor the subprocesses
started. The default functionality of subprocess.Popen only gives the output of
the process after it has terminated, which is obviously not what we want.
Optionally, pipes can be used instead. Here, glibc poses a problem because it
automatically starts output buffering when it detects it is writing to a pipe.
Therefore we use a pty (pty.openpty()) instead to obtain a pseudo terminal, on
which no output buffering is performed. The ptys are wrapped in the ThreadedPipe
object which reads all output and errors and passes it to the GUI.
\end{document}
